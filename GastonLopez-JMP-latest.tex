
% import my template for papers
\documentclass{gaston-paper-template}

% Choose wheter to compile just the intro (1) intro + sections (2) or intro + sections + appendix (3)
\newcounter{subfilecounter}
\setcounter{subfilecounter}{3}


% Choose spec number for results 
\newcommand{\specnumber}{4}

% Choose statistic for graphs and tables (mean, median, etc)
\newcommand{\mean}{weighted_average}

  % Set the counter to 1

\title{ \vspace*{-2.5cm} \hspace*{-0.5cm}How Do Governments Engage in Price Discrimination? Evidence from a Large-Scale Nationalization \footnote{
    We thank Vivek Bhattacharya, David Dranove, Robert Porter, Gast\'on Illanes, Igal Hendel, Nicola Persico, Mar Reguant, Bill Rogerson, Edoardo Tesso, Anran Li, Tomas Wilner, Jingyuan Wang, Diego Huerta, Matthew O'Keefe, Ignacio Caro Solis and Julian Rojo for helpful comments. We also thank seminar participants at the Industrial Organization Lunch.
}}


\author{Gast\'on L\'opez (JMP) \thanks{Northwestern University} 
\and Francisco Pareschi \thanks{Northwestern University}}

\date{\today}


\begin{document}

\begin{singlespace}
\maketitle 


\centering 
\href{https://gaston-lopez.github.io/GastonLopez-JMP-latest.pdf}{Click here to access the latest version}.


\begin{abstract}
    State-owned enterprises (SOEs) have the potential to correct market failures, but they are also subject to the influence of politics and interest groups. We examine this trade-off in the context of the nationalization of the leading gasoline company in Argentina. Descriptive analysis suggests that pricing patterns changed after the nationalization. First, the government exerted less market power, charging lower prices on average and benefiting consumers. Second, it engaged in less \emph{economic price discrimination}, reducing the correlation between prices and consumers' willingness to pay. Third, it engaged in \emph{political price discrimination}, charging lower prices in provinces with political connexions with the state-owned firm. Motivated by these findings, we develop and estimate a model of gasoline supply and demand under market power and recover the government's objective function. We find that public provision lead to welfare gains but is also associated with redistributive motives. Compared to a benevolent planner that internalizes the welfare of all consumers and firms equally, the government set prices as if it only cares about favoring middle-income consumers and consumers in provinces that have political ties with the firm. Lastly, we use the model to assess the company’s response to policy alternatives, including pricing rules that align government actions with the public interest and are in place in government agencies worldwide. Our findings show that rules effectively reduce the influence of politics in pricing but are associated with higher costs: they mitigate half of the welfare gains generated by the nationalization and increase the taxpayers' burden by 10\%. These findings emphasize the importance of politics and interest groups in shaping governments' decision-making process and the role of SOEs as instruments for redistribution.
    \end{abstract}
\end{singlespace}

% \begin{quote}
%     \emph{“When corporate interests are not aligned with national interests, when companies are concerned only with profits, that’s when economies fail"} Cristina Fernandez, Argentinan President \footnote{https://www.reuters.com/article/us-argentina-ypf-idUSBRE8421GV20120504}    
% \end{quote}
    


\thispagestyle{empty}


% \begin{quote}
%     \emph{“When corporate interests are not aligned with national interests, when companies are concerned only with profits, that’s when economies fail"} 
%     \end{quote}
    
%     \indentthis \indentthis Cristina Fernandez, Former Argentinan President \footnote{See "Argentina nationalizes oil company YPF," Name of Newspaper, May 3, 2012, \url{https://www.reuters.com/article/us-argentina-ypf-idUSBRE8421GV20120504}}

% \clearpage

\newpage
\setcounter{page}{1}

\ifnum\value{subfilecounter}>0
    \subfile{section_1_intro/introduction.tex}
\fi


\ifnum\value{subfilecounter}>1
    %% SECTION 2 -- background 
    %\newpage
    \subfile{sections_2_and_3/section_2.tex}
    % % SECTION 3 -- DESCRIPTIVES
    % %\newpage
   \subfile{section_3/section_3.tex}
    % %% SECTION 4 Model 
     %\newpage 
   \subfile{section_4_model/section_4_model.tex}
    % %% SECTION 5 -- Results
    % %\newpage
   \subfile{section_4_model/section_4_results/section_4_results.tex}

    %% SECTION 6 -- Effects
   \subfile{section_4_model/section_5.tex}



    %% SECTION 7 -- Counterfactuals
    %\newpage
    \subfile{section6_counterfactuals/section6.tex}

    %% SECTION 8 -- Conclusion
   % \newpage
  \subfile{section7_conclution/conclu.tex}
\fi

\newpage
% Bibliography
%\printbibliography

\bibliographystyle{plainnat}
\bibliography{sample.bib}

% %% Appendix

\ifnum\value{subfilecounter}>2

    \newpage
    \appendix

    % %% Appendix sections 2 
    % %\newpage
   \subfile{sections_2_and_3/appendix_S2/appendix_s2}

    % Appendix sections 3
%    \newpage
   \subfile{section_3/appendix/appendixS3.tex}



    % %% Appendix sections 5 - results 
    \subfile{section_4_model/section_4_results/appendix/appendixS5_results}

    %% markups appendix
    \subfile{section_4_model/section_4_results/appendix_identificaiton/appendix_markups.tex}
    % %% Appendix sections 4 and 5
    %%%%%%%%%%\subfile{section_4_model/appendices/AS4.tex}


    % % %\newpage
    \subfile{appendix_estimation/appendix_estimation}

    % % %% Appendix sections 3
%    \newpage

    % %% Appendix sections 6
    \subfile{section6_counterfactuals/appendix/appendixs6.tex}

    %% Appendix sections 7
    \subfile{section6_counterfactuals/appendix/appendixsec7.tex}

    % %% Robustness checks
    % %\subfile{/appendix_robustness/robustness}

\fi



\end{document}